\documentclass{beamer}

% \makeatletter
%   \def\beamer@calltheme#1#2#3{%
%     \def\beamer@themelist{#2}
%     \@for\beamer@themename:=\beamer@themelist\do
%     {\usepackage[{#1}]{\beamer@themelocation/#3\beamer@themename}}}
% 
%   \def\usefolder#1{
%     \def\beamer@themelocation{#1}
%   }
%   \def\beamer@themelocation{}
% \makeatother

% \usefolder{Theme}
\usetheme{NLeSC}
\usepackage{tikz}
% \usepackage{grffile}
\usetikzlibrary{arrows,shapes}
\tikzstyle{every picture}+=[remember picture]
\everymath{\displaystyle}
\tikzstyle{na} = [baseline=-.5ex]


\title{Kernel Estimation for a Non-Parametric Cointegrating Regression Model}
\author{Derek Feng}
\date{October 21st, 2011}

\AtBeginSection{
\begin{frame}
\begin{center}
\structure{\Huge \insertsection}
\end{center}
\end{frame}
}

\begin{document}




\frame{
  \titlepage
}

\frame{
 \frametitle{Outline}
 \tableofcontents
}

  

\section{Introduction} % (fold)

\subsection{Motivation} % (fold)

\begin{frame}{Question}
  \begin{center}
    \structure{How do we model this relationship?}
  \end{center}
\end{frame}

\begin{frame}{Solution?}
  \begin{center}
    Ordinary Least Squares Regression
  \end{center}
  \pause
  \begin{align*}
    y_t = \beta_0 + \beta_1 x_t + \epsilon_t
  \end{align*}
\end{frame}

\begin{frame}[c]\frametitle{Results of performing an OLS regression}
  \begin{tabular}{cccccc}
  \hline
   & Estimate & Std. Error & t value & p value & \\
  \hline
  $\beta_0$ & 1.595 & 0.526 & 3.028 & 0.0028 & \textcolor<2>{red!70!bg}{**}\\
  \hline
  $\beta_1$ & 1.044 & 0.065 & -16.2 & $<2\text{e}16$ & \textcolor<2>{red!70!bg}{***}\\
  \hline
  \end{tabular}

  Multiple R-squared: 0.5698, Adjusted R-squared: 0.5676
\end{frame}

\begin{frame}
  \begin{center}
    {\Huge no}
    \pause
    
    \vspace{2cm}
    
    They were \alert{independent} random walks
  \end{center}
\end{frame}


\begin{frame}[t]\frametitle{Why OLS doesn't hold with nonstationary processes}
  
  \begin{center}
    \alert{
    \uncover<2,4>{nonstationary}
    \uncover<4>{$\neq$}
    \uncover<3,4>{stationary}
    } 
  \end{center}
  
  \begin{equation*}
    \alert<2>{y_t -\beta_0 + \beta_1 x_t} = \alert<3>{\epsilon_t}
  \end{equation*}
\end{frame}

% \begin{frame}[c]\frametitle{Spurious Regression}
%   In the limit as $n \rightarrow \infty$,
%   \begin{itemize}
%     \item<2-> The OLS estimator of $\beta_1$ does not converge to zero
%     \item<3-> The OLS $t$-statistic for $\beta_1$ diverge to $\pm \infty$
%     \item<4-> $R^2 \rightarrow 1$
%   \end{itemize}
% \end{frame}

\begin{frame}
  \begin{center}
    {\Large however \dots}
  \end{center}
\end{frame}

\begin{frame}[t]\frametitle{What if \dots}
  \begin{equation*}
    \alert{y_t -\beta_0 + \beta_1 x_t} = \epsilon_t
  \end{equation*}
\end{frame}

\begin{frame}[c]\frametitle{Shortcomings}
  % \begin{center}
  %   Nonlinear Economic Model
  %   % \uncover<2->{: Production Theory} 
  % \end{center}
    \begin{itemize}
      \item<1-> Nonlinearity in economic processes
      \item<3-> Relationship unchanged over an extended period of time
    \end{itemize}
    
  \uncover<2>{\begin{example}{The Cobb Douglas production function}
    \begin{equation*}
      y = \phi x_1^{\alpha} x_2^{\beta}
    \end{equation*}
    where $y$ is the total production, $x_1$ is labour input and $x_2$ is capital input.
  \end{example}}
\end{frame}

% THERE ARE A PLETHORA OF NONLINEAR MODELS

\begin{frame}[c]\frametitle{Nonlinear cointegration model}
  \begin{align*}
    y_t = f(x_t) + \text{error}
  \end{align*}
  \begin{itemize}
    \item $y_t, x_t$ nonstationary
    \item $f$ nonlinear
    \item \alert{stationary} error 
  \end{itemize}
  % where the error is \alert{stationary}.
  
\end{frame}

\begin{frame}[c]\frametitle{Question}
  \begin{center}
    What is $f$?
  \end{center}
\end{frame}

\begin{frame}[c]\frametitle{Parametric vs Non-Parametric}
  \begin{columns}
    \begin{column}{0.5\textwidth}
      Parametric
      \begin{itemize}
        \item<2-> Misspecification
      \end{itemize}
    \end{column}
    \begin{column}{0.5\textwidth}
      \uncover<3->{Non-Parametric}
      \begin{itemize}
        \item<4-> Let the data speak for itself
      \end{itemize}
    \end{column}
  \end{columns}
\end{frame}

\begin{frame}[c]\frametitle{Nonlinear + Nonstationary}
  \begin{center}
    \begin{tabular}{ccc}
     & Stationary & Nonstationary\\
    Linear & -- & --\\
    Nonlinear & -- & \checkmark\\
    \end{tabular}
  \end{center}
\end{frame}


\begin{frame}[c]\frametitle{Difficulties}
  Nonstationary + Nonparametric =\alert{
  \only<1>{?}\only<2,3>{wandering + }\only<3>{local behaviour}\only<4>{Difficult}\only<5>{Reduced rate of convergence}\only<6>{New techniques required}
  }
\end{frame}

\begin{frame}[t]\frametitle{Local Time}
  \begin{definition}[Local Time]
    The local time up to time $t$ at $x$ is
    \begin{align*}
      L(t,x) = \lim_{\epsilon \downarrow 0} \frac{1}{2 \epsilon} \int_{0}^{t} I(|B_s-a|) ds
    \end{align*}
    \pause
    \alert{$L(t,x)$ is the amount of time that the path of $B_t$ spends at $x$ up to time $t$}
  \end{definition}
\end{frame}

\begin{frame}[c]\frametitle{Kernel Regression}
  the Nadaray-Watson Kernel Estimator:
  \begin{align*}
    \hat{f}(x) = \frac{\sum_{t=1}^{n} y_t K_h(x_t-x)}{\sum_{t=1}^{n} K_h(x_t-x)} 
  \end{align*}
  where
  \begin{align*}
    K_h(x) = \frac{1}{h} K(x/h)
  \end{align*}
  \pause
  \begin{block}{Note}
    $h$ controls the size of the neighbourhood, and the rate at which $K_h$ decays
  \end{block}
\end{frame}

\begin{frame}[t]{The Equation}
  
  \begin{equation*}
    y_t
    = 
    f(x_t) + u_t
  \end{equation*}
  
  \begin{itemize}
    \item $x_t = \sum_{j=1}^{t} \epsilon_j,\quad \epsilon_{j}\sim iid(0,1)$
    \item Nonparametric estimator of $f$: NW Kernel Estimator.
    \item \alert{$u_t = ?$}
    % \item \alert{The choice of $u_t$ is important - depends on environment.}
  \end{itemize}

\end{frame}

\begin{frame}[c]\frametitle{Previous work}
  \begin{itemize}[<+->]
    \item Wang, Phillips (2009): error process as a \alert{martingale difference sequence}
    \item We consider the error process as a \alert{linear process} 
  \end{itemize}
\end{frame}

\begin{frame}[t]{Linear Process}
  \begin{definition}[Linear Process]
    Let $u_t$ be defined by \[ u_t = \sum_{j=0}^{\infty} \phi_j \epsilon_{t-j}, \quad t \geq 1 \] where $\{\epsilon_i\}$ is a sequence of iid random variables with mean $0$ and variance $1$.
  \end{definition}
  \begin{itemize}
    \item<2-> linear aggregation of random shocks
  \end{itemize}
\end{frame}

\begin{frame}[c]\frametitle{Methodology}
  \begin{align*}
    y_t = f(x_t) + u_t
  \end{align*}
  Recall the kernel estimator:
  \begin{align*}
    \hat{f}(x) = \frac{\sum_{t=1}^{n} y_t K_h(x_t-x)}{\sum_{t=1}^{n} K_h(x_t-x)} 
  \end{align*}
\end{frame}
\begin{frame}[c]\frametitle{Methodology}
  \begin{equation*}
    \hat{f}(x) - f(x) = \frac{\alert<3,5>{\sum_{t=1}^n u_t K_h(x_t - x)}}{\alert<3,4>{\sum_{t=1}^n K_h(x_t - x)}} + \alert<2>{\frac{\sum_{t=1}^n (f(x_t)-f(x)) K_h(x_t - x)}{\sum_{t=1}^n K_h(x_t - x)}}
  \end{equation*}
\end{frame}


% section (end)

\end{document}